\section{Translating Obolog to OWL}

We provide 3 interpretations for the logical semantics of Obolog.

\begin{clist}
\item A standard/normative OWL-DL translation, which makes use of ``hidden'' time-slices to get around the lack of n-ary relations in OWL
\item A simplied OWL-DL translation
\item An OWL-Full (RDFS) translation, in which types (classes) and type level relations are in the domain of discourse
\end{clist}

The normative translation is non-trivial. It attempts to do justice to
the use of 3-ary relations in defining binary type-level relations. It
is predicated on the widely accepted assumption that Continuants
(entities exist in whole in any point in time, but can gain or lose
parts through time) are difficult to deal with in OWL due to all
relations (properties) being binary.

The translation works within this constraint, and provides an
interpretation of BFO/OWL in which references to continuant types are
translated as references to corresponding time-slices.

In addition, we provide an extension translation for obolog-A, and a
translation for obolog-lex in terms of OWL Annotation Properties.

\subsection{Standard DL Translation}

\input{../owl/obolog-owl-dl-lp}

\subsection{Simplified Translation (not normative)}

\input{../owl/obolog-owl-simple-lp}

\subsection{OWL Full (RDFS) Translation}

\input{../owl/obolog-rdfs-lp}

\subsection{Obolog-lex Translation}

Annotation Properties

\subsection{Obolog-A Translation}

TODO

OWL2 features?