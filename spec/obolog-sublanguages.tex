\def\ObologCL{\pr{obolog$^{CL}$}}
\def\ObologCore{\pr{obolog$^{Core}$}}
\def\ObologACore{\pr{obolog$^{A/Core}$}}
\def\ObologA{\pr{obolog$^{A}$}}
\def\ObologH{\pr{obolog$^{H}$}}
\def\ObologData{\pr{obolog$^{Data}$}}
\def\CL{\pr{CL}}
\def\IKL{\pr{IKL}}

\section{Obolog Sublanguages and Superlanguages}

This section is incomplete

\subsection{\CL}

$\CL$ is the set of all common logic texts

\subsection{\IKL}

$\IKL$ is the set of all IKL texts. Every \CL\ text is an \IKL\ text.

$\CL \subset IKL$

\subsection{\ObologCL}

Any CL text that imports the Obolog predicate definitions is an \ObologCL\ text

$\ObologCL \subset \CL$

\subsection{\ObologCore}

Any \ObologCL\ text that does not use quantified sentences (excluding
the CL axioms for the Obolog predicates themselves).

\begin{eqnarray*}
\ObologCore \subset \ObologCL
\end{eqnarray*}

The suffix ``Core'' denotes a subset that excludes arbitrary quantified sentences.

\subsection{\ObologA}

A subset of \IKL\ in which the \pr{that} function is used for atomic sentences only. 

Every \ObologA\ text is an \IKL\ text:
\begin{eqnarray*}
\ObologA \subset \IKL
\end{eqnarray*}

Every \ObologCL\ text is an \ObologA\ text
\begin{eqnarray*}
\ObologCL \subset \ObologA
\end{eqnarray*}

Again we can define \ObologACore\ as the subset of \ObologA that
excluded quantified sentences.

\begin{eqnarray*}
\ObologACore = \ObologA \cap \ObologCore
\end{eqnarray*}

\subsection{\ObologH\ and \ObologData}

\subsubsection{\ObologH}

Horn clause subset of \ObologCL\. It extends \ObologCore\ by allowing
certain quantified sentences, specifically those correponding to horn rules.

\begin{eqnarray*}
\ObologH \subset \ObologCL
\end{eqnarray*}

%\ObologCoreH \subset \ObologCore \\
%\ObologCoreH \subset \ObologH


\subsubsection{\ObologData}

\begin{eqnarray*}
\ObologData \subset \ObologH
\end{eqnarray*}

Datalog subset of \ObologCL\ i.e. only atomic sentences, or
universally quantified sentences with one term on the LHS, and no
function symbols.  Query evaluation with Datalog is sound and complete
and can be done efficiently even for large texts (databases).

%Note: The Obo-Edit reasoner uses a subset of Obolog-Data-Core (no
%arbitrary horn rules) dealing with types only, augmented with
%constraints (e.g. for checking disjointness violations).

\subsubsection{Horn Rules}

either restatements of obolog axioms as horn rules, or rules derived from axioms + RO definitionsRule: reflexivity of is\_a

\begin{eqnarray*}
 \pr{type}(x) &\imp & \pr{is\_a}(x, x) 
\end{eqnarray*}
Rule: transitivity of is\_a

\begin{eqnarray*}
 \pr{is\_a}(a, b) \con \\
 \pr{is\_a}(b, c) &\imp & \pr{is\_a}(a, c) 
\end{eqnarray*}
Rule: equivalence if mutual is\_a

\begin{eqnarray*}
 \pr{is\_a}(a, b) \con \\
 \pr{is\_a}(b, a) &\imp & \pr{equiavelent\_to}(a, b) 
\end{eqnarray*}
Rule: equivalence if mutual is\_a, inv

\begin{eqnarray*}
 \pr{equiavelent\_to}(a, b) &\imp & \pr{is\_a}(a, b) 
\end{eqnarray*}
Rule: transitivity of subrelation

\begin{eqnarray*}
 \pr{subrelation}(a, b) \con \\
 \pr{subrelation}(b, c) &\imp & \pr{subrelation}(a, c) 
\end{eqnarray*}
XRule: propagation over/under is\_a for all-some relations

\begin{eqnarray*}
 \pr{is\_a}(a, b) \con \\
r(b, c) \con \\
 \pr{is\_a}(c, d) \con \\
 \E ri[ \pr{all\_some}(r, ri) ]&\imp &r(a, d) 
\end{eqnarray*}
XRule: propagation over/under is\_a for all-only relations

\begin{eqnarray*}
 \pr{is\_a}(a, b) \con \\
r(b, c) \con \\
 \pr{is\_a}(c, d) \con \\
 \E ri[ \pr{all\_only}(r, ri) ]&\imp &r(a, d) 
\end{eqnarray*}
XRule: all-only constraints on subtypes

\begin{eqnarray*}
 \pr{all\_only}(ru, ri) \con \\
 \pr{all\_some}(re, ri) \con \\
ru(b, y) \con \\
re(a, x) \con \\
 \pr{is\_a}(a, b) &\imp & \pr{is\_a}(x, y) 
\end{eqnarray*}
XRule: transitivity

\begin{eqnarray*}
 \pr{transitive}(r) \con \\
r(a, b) \con \\
r(c, c) &\imp &r(a, c) 
\end{eqnarray*}
XRule: inverse\_of

\begin{eqnarray*}
 \pr{inverse\_of}(r, s) \con \\
r(a, b) &\imp &r(b, a) 
\end{eqnarray*}
Rule: symmetricality of inverse\_of

\begin{eqnarray*}
 \pr{inverse\_of}(r, s) &\imp & \pr{inverse\_of}(s, r) 
\end{eqnarray*}
Rule: symmetricality of disjoint\_from

\begin{eqnarray*}
 \pr{disjoint\_from}(r, s) &\imp & \pr{disjoint\_from}(s, r) 
\end{eqnarray*}
XRule: subrelations

\begin{eqnarray*}
r(a, b) \con \\
 \pr{subrelation}(r, s) &\imp &s(a, b) 
\end{eqnarray*}
XRule: transitive\_over

\begin{eqnarray*}
 \pr{transitive\_over}(r, over) \con \\
r(a, b) \con \\
over(b, c) &\imp &r(a, c) 
\end{eqnarray*}
XRule: holds\_over\_chain

\begin{eqnarray*}
 \pr{holds\_over\_chain}(r, r1, r2) \con \\
r1(a, b) \con \\
r2(b, c) &\imp &r(a, c) 
\end{eqnarray*}
XRule: cyclic

\begin{eqnarray*}
r(a, b) \con \\
r(b, a) \con \\
 \n(a = b)&\imp & \pr{cyclic}(r) 
\end{eqnarray*}
XRule: functional relations

\begin{eqnarray*}
 \pr{functional}(r) \con \\
r(a, b) \con \\
r(a, c) &\imp & \pr{equivalent\_to}(b, c) 
\end{eqnarray*}
XRule: symmetricality

\begin{eqnarray*}
 \pr{symmetric}(r) \con \\
r(a, b) &\imp &r(b, a) 
\end{eqnarray*}
XRule: domain

\begin{eqnarray*}
 \pr{domain}(r, x) \con \\
r(a, b) &\imp & \pr{instance\_of}(a, x) 
\end{eqnarray*}
XRule: range

\begin{eqnarray*}
 \pr{range}(r, x) \con \\
r(a, b) &\imp & \pr{instance\_of}(b, x) 
\end{eqnarray*}
\begin{eqnarray*}
 \pr{all\_only}(ru, ri) \con \\
ru(X, Y) \con \\
ri(a, b) \con \\
 \pr{instance\_of}(a,  \pr{Occurrent}) \con \\
 \pr{instance\_of}(b,  \pr{Occurrent}) \con \\
 \pr{instance\_of}(a, X) &\imp & \pr{instance\_of}(b, Y) 
\end{eqnarray*}
Constraint: disjoint pairs share no is\_a children

\begin{eqnarray*}
 \pr{is\_a}(a, x) \con \\
 \pr{is\_a}(b, x) \con \\
 \pr{disjoint\_from}(a, b) &\imp & \pr{unsatisfiable}(a) 
\end{eqnarray*}









