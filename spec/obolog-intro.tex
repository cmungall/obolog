\section{Introduction}

OBO-Format is an exchange format for ontologies and associated
information. This document provides a formalization of both the syntax
and semantics of OBO-Format version 1.3.

The syntax is specified as a BNF grammar. The semantics are defined in
terms of a translation to first order logic - more specifically, to
the Obolog language, a collection of constructs defined using the
Common Logic ISO standard.

\subsection{Background}

OBO Format 1.0 was devised in 2001, originally as a successor to the
original Gene Ontology DAG file format. The original design was
informed by local pragmatic considerations, and sacrified syntactic
and semantic rigour in favour of simplicity. OBO Format conceives of
ontologies in graph-oriented terms, with each node of the graph
described using a set of tag-value pairs.

OBO Format 1.2. was initiated in 2003 finalized in 2006, and extended
1.0 with tags borrowed from the OWL (originally DAML+OIL)
language. In 2007 Ian Horrocks suggested a formalization of both syntax and
semantics in terms of OWL1 and OWL2.

\subsection{Core Concepts}

\subsubsection{Types, instances, relations and annotations}

The universe of discourse of OBO consists of entities, which are
divided into types, instances, relations and annotations. This
division is exhaustive and pairwise disjoint (i.e. every entity is
exactly one of these things).

Instances are spatiotemporal particulars, whereas types are patterns
which are shared by collections of like instances. Types are denoted
by terms such as ``Lung'' or ``Apoptosis''.

Relations obtain between entities. In OBO, relations can hold between
entities of any kind (even relations). If a relation \pr{Rel} holds
between two or more entities we write this as a tuple of \pr{Rel}:

\begin{verbatim}
Rel(Argument1, Argument2, ... ArgumentN)
\end{verbatim}

Where \pr{Rel} is drawn from a collection of relations, and
Arguments 1 to N are drawn from the collection of entities.

In cases where the tuple has a single argument, we conventionally call
the relation a \emph{unary predicate}.

\subsubsection{Ontology Graphs}

Note that whilst tuples can have any number of arguments, OBO is
geared towards relations that either take 2 or 3 arguments (i.e. 2-ary
and 3-ary relations). These relations are conceived of in
graph-theoretic terms, with each entity constituting a node in the
graph, and each tuple consituting an egde, between the first argument
(called the ``child'' or ``subject'') and the second argument (called
the ``parent'', ``object'' or ``target''). The relation acts as
edge-label, with additional arguments as edge properties.

\subsection{Formalization}

This specification provides the normative formalization of OBO-Format
in terms of a language called \emph{Obolog}. Obolog is a collection of
predicates and functions with formally defined semantics that can be
used to express ontology graphs.

\begin{clist}

\item Every OBO-Format document has a normative interpretation as an Obolog
text. Obolog is expressed using Common Logic (CL) [TODO: currently
using KIF, which can be automatically translated to CLIF, a CL
syntax].

\item Every Obolog text is a CL text(s?).

\item Obolog can be used independently of OBO-Format, any CL syntax can be
used.

\end{clist}

We distinguish between two sub-languages: Obolog-core and
Obolog-full. An Obolog-core text is a CL text that contains no
quantified sentences (possibly other restrictions?).  Obolog-full is
Obolog-core extended with arbitrary quantified sentences. We
distinguish between trivial and non-trivial membership in these
classes depending on whether the text contains any sentences using
Obolog predicates.

\subsubsection{Translation to other FOL syntaxes}

CL is a highly expressive language for First Order Logic. Other FOL
syntaxes may disallow certain constructs: for example, variable as
first argument in a sentence.

It is possible to translate to other syntaxes, such as Prover9 syntax,
by treating the definitions of Obolog predicates as macro-expansion
rules. For example, The Obolog definition of transitivity for binary
relations is as follows:

\begin{eqnarray*}
 \pr{transitive}(rel) \con \\
rel(X, Y) \con \\
rel(Y, Z) &\imp &rel(X, Z) 
\end{eqnarray*}

We can use this to expand the sentence $transitive(develops\_from)$ to:

\begin{eqnarray*}
develops\_from(X, Y) \con \\
develops\_from(Y, Z) &\imp develops\_from(X, Z) 
\end{eqnarray*}

In future we may define sublanguages based on FOL profile; for example
Obolog-Horn for Horn rules, which is useful for rule systems and
relational databases.

\subsubsection{Translation to OWL}

Previous attempts to formalize OBO-Format in terms of OWL did so via a
direct translation of OBO-Format. This can now be done in terms of Obolog.

\subsubsection{The Relation Ontology and The Basic Formal Ontology}

We attempt to be as ontologically neutral as possible. Certain
ontological assumptions are assumed: a division between Continuants
and Occurrents. 

