\section{Relations from GO/OBO}
These declarations and
definitions are imported from the OBO Relation Ontology. We only use
the is\_a and part\_of relations here, and only show a subset of the
axioms of relevance for phylogenetic functional inference.
\subsubsection{Relation:  \pr{is\_a}}


Formal definition from the RO: For continuants:
C is\_a C' if and only if: given any c that instantiates C at a time t,
c instantiates C' at t. For processes: P is\_a P' if and only if: that
given any p that instantiates P, then p instantiates P'.

Gene annotations propagate over is\_a

\emph{Axiom: the is\_a relation is transitive}

\begin{eqnarray*}
 \pr{transitive}( \pr{is\_a}) 
\end{eqnarray*}
\subsubsection{Relation:  \pr{part\_of}}


For continuants, we have C part\_of C1 if and only if any instance of C at any time is an instance-level part of some instance of C1 at that time, as for example in: cell nucleus part\_ of cell.

Formal definition from the RO: Gene annotations propagate over part\_of

\emph{Axiom: the part\_of relation is transitive}

\begin{eqnarray*}
 \pr{transitive}( \pr{part\_of}) 
\end{eqnarray*}
\section{Phylogenetic Tree Relations and Axioms}
Axioms for
Phylogenetic Trees. A tree consists of nodes connected in
directly\_descended\_from relationships. We infer descended\_from
relationships from these. Here we assume that no gene can be a node in
more than one phylogenetic tree, so we conflate gene and node
together. In particular, intermediate nodes correspond to ancestral
genes.
\subsubsection{Relation:  \pr{directly\_descended\_from}}


X is directly\_descended\_from
Y if and only X has an immediate parent Y in a phylogenetic
tree.

These relationships are not inferred
within this framework. It is assumed we already have a single unified
tree with directly\_descended\_from relationships.

\subsubsection{Relation:  \pr{descended\_from}}


\begin{eqnarray*}
 \pr{transitive}( \pr{descended\_from}) 
\end{eqnarray*}
\begin{eqnarray*}
 \pr{irreflexive}( \pr{descended\_from}) 
\end{eqnarray*}
X is descended\_from Y if and only 
X has an immediate parent Y in a phylogenetic tree. The notion of
'direct' descendant is artificial here as it depends on how many
sequences were aligned.

\emph{Axiom: x is descended from y if x is directly descended from y. This is the base case for the recursive axiom below.}

\begin{eqnarray*}
 \pr{directly\_descended\_from}(x, y) &\imp & \pr{descended\_from}(x, y) 
\end{eqnarray*}
\emph{Axiom: x is descended from y if x is directly descended from z and z is descended\_from z. This is the recursive form of descended\_from.}

\begin{eqnarray*}
 \pr{directly\_descended\_from}(x, z) \con \\
 \pr{descended\_from}(z, y) &\imp & \pr{descended\_from}(x, y) 
\end{eqnarray*}
\subsubsection{Relation:  \pr{common\_ancestor}}


\begin{eqnarray*}
 \pr{descended\_from}(x, a) \con \\
 \pr{descended\_from}(y, a) &\imp & \pr{common\_ancestor}(x, y, a) 
\end{eqnarray*}
\subsubsection{Relation:  \pr{least\_common\_ancestor}}


\begin{eqnarray*}
 \pr{common\_ancestor}(x, y, a) \con \\
 \n( \E b[ \pr{common\_ancestor}(x, y, b) \con \\
 \pr{descended\_from}(b, a) ])&\imp & \pr{least\_common\_ancestor}(x, y, a) 
\end{eqnarray*}
\subsubsection{Unary predicate:  \pr{leaf\_node}}


a node without descendants in the phylogenetic tree.

\emph{Axiom: x is a leaf\_node if and only if x has no descendants.}

\begin{eqnarray*}
 \pr{leaf\_node}(x) &\dimp & \n( \E y[ \pr{descended\_from}(x, y) ])
\end{eqnarray*}
\subsubsection{Unary predicate:  \pr{root\_node}}


x is a root\_node if and only if x is not descended from anything

\emph{Axiom: x is a root\_node if and only if x has no ancestors.}

\begin{eqnarray*}
 \pr{root\_node}(x) &\dimp & \n( \E y[ \pr{descended\_from}(y, x) ])
\end{eqnarray*}
\subsubsection{Relation:  \pr{nad}}


x nad y if and only if x and y do not stand in a descended\_from relationship to the other. i.e. they are neither ancestors nor descendants of each other

\begin{eqnarray*}
 \pr{exact\_synonym}( \pr{nad},  \pr{not-ancestor-or-descendant}) 
\end{eqnarray*}
\emph{Axiom: every node stands in a nad relation to itself (because nothing descends from itself)}

\begin{eqnarray*}
 \pr{reflexive}( \pr{nad}) 
\end{eqnarray*}
\emph{Axiom: x nad y if and only if x is not descended from y and y is not descended from x}

\begin{eqnarray*}
 \n( \pr{descended\_from}(x, y) )\con \\
 \n( \pr{descended\_from}(y, x) )&\imp & \pr{nad}(x, y) 
\end{eqnarray*}
\section{Basic Functional Inference Axioms}
Axioms for inferring
function for genes based on the ontology. These are independent of
phylogenetic trees. We divide these into axioms for POSITIVE
annotation statements (has\_function) and NEGATIVE annotations
statements (lacks\_function). Note that here when we talk of function,
we are talking broadly of the functional role within a larger
biological\_process, and the subcellular site at which the function is
realized.
\subsection{Positive Inference}
Relationships stating POSITIVE
annotations, and axioms for inferences upon them.
\subsubsection{Relation:  \pr{asserted\_has\_function}}


G asserted\_has\_function if and
only if G has function F, and a statement asserting this fact has been
made by a curator with supporting evidence. This could come from Gene
Association Files, excluding NOT annotations

\begin{eqnarray*}
 \pr{comment\_has\_function}( \pr{note that here function is used in a general
sense covering GO annotations to molecular\_function,
biological\_process and cellular\_component. Saying G has\_function
Peroxisome is shorthand for saying G encodes a gene product that
realizes its function within the Peroxisome}) 
\end{eqnarray*}
Example: \begin{eqnarray*}
 \pr{asserted\_has\_function}( \pr{HSA:P53},  \pr{GO:DNA\_Repair}) 
\end{eqnarray*}
\subsubsection{Relation:  \pr{has\_function}}


\emph{Axiom: G has\_function F if a curator says so (with evidence). This forms the base case in recursive inference}

\begin{eqnarray*}
 \pr{asserted\_has\_function}(g, f) &\imp & \pr{has\_function}(g, f) 
\end{eqnarray*}
Example: \begin{eqnarray*}
 \pr{asserted\_has\_function}( \pr{HSA:P53},  \pr{GO:DNA\_Repair}) &\imp & \pr{has\_function}( \pr{HSA:P53},  \pr{GO:DNA\_Repair}) 
\end{eqnarray*}
\emph{Axiom: G has\_function F if G has\_function F' and F' is a kind of F. Note this is a recursive rule.}

\begin{eqnarray*}
 \pr{has\_function}(g, f_2) \con \\
 \pr{is\_a}(f_2, f) &\imp & \pr{has\_function}(g, f) 
\end{eqnarray*}
Example: \begin{eqnarray*}
 \pr{has\_function}( \pr{HSA:P53},  \pr{GO:DNA\_Repair}) \con \\
 \pr{is\_a}( \pr{GO:DNA\_Repair},  \pr{GO:DNA\_Metabolism}) &\imp & \pr{has\_function}( \pr{HSA:P53},  \pr{GO:DNA\_Metabolism}) 
\end{eqnarray*}
Example: \begin{eqnarray*}
 \pr{has\_function}( \pr{HSA:P53},  \pr{GO:DNA\_Metabolism}) \con \\
 \pr{is\_a}( \pr{GO:DNA\_Repair},  \pr{GO:Metabolism}) &\imp & \pr{has\_function}( \pr{HSA:P53},  \pr{GO:Metabolism}) 
\end{eqnarray*}
\emph{Axiom: G has\_function F if G has\_function F' and F' is part of F. Note this is a recursive rule.}

\begin{eqnarray*}
 \pr{has\_function}(g, f_2) \con \\
 \pr{part\_of}(f_2, f) &\imp & \pr{has\_function}(g, f) 
\end{eqnarray*}
Example: \begin{eqnarray*}
 \pr{has\_function}( \pr{MGI:Pex1},  \pr{GO:Peroxisome}) \con \\
 \pr{part\_of}( \pr{GO:Peroxisome},  \pr{GO:Cytoplasm}) &\imp & \pr{has\_function}( \pr{MGI:Pex1},  \pr{GO:Cytoplasm}) 
\end{eqnarray*}
\subsection{Negative Inference}
Axioms for making inferences based
on negative annotations. Note that annotations here propagate DOWN
the GO graph
\subsubsection{Relation:  \pr{asserted\_lacks\_function}}


G asserted\_lacks\_function if
and only if G lacks function F, and a statement asserting this fact
has been made by a curator with supporting evidence. This would come
from lines with operator NOT in Gene Association Files

\subsubsection{Relation:  \pr{lacks\_function}}


A relation between a gene F and a function F, such that G does not have function F

\emph{Axiom: G lacks\_function F if and only if not( G has\_function F )}

\begin{eqnarray*}
 \pr{lacks\_function}(g, f) &\dimp & \n( \pr{has\_function}(g, f) )
\end{eqnarray*}
\emph{Axiom: G lacks\_function F if a curator says so}

\begin{eqnarray*}
 \pr{asserted\_lacks\_function}(g, f) &\imp & \pr{lacks\_function}(g, f) 
\end{eqnarray*}
\emph{Axiom: downward propagation of negated annotations over is\_a}

\begin{eqnarray*}
 \pr{lacks\_function}(g, f_2) \con \\
 \pr{is\_a}(f, f_2) &\imp & \pr{lacks\_function}(g, f) 
\end{eqnarray*}
\emph{Axiom: downward propagation of negated annotations over part\_of}

\begin{eqnarray*}
 \pr{lacks\_function}(g, f_2) \con \\
 \pr{part\_of}(f, f_2) &\imp & \pr{lacks\_function}(g, f) 
\end{eqnarray*}
Example: \begin{eqnarray*}
 \pr{lacks\_function}( \pr{P35431},  \pr{GO:Mammary\_gland\_development}) \con \\
 \pr{part\_of}( \pr{GO:Lactation},  \pr{GO:Mammary\_gland\_development}) &\imp & \pr{lacks\_function}( \pr{P35431},  \pr{GO:Lactation}) 
\end{eqnarray*}
\emph{Axiom: G does not have function F if G is in an organism lacking that function}

\begin{eqnarray*}
 \pr{in\_organism\_type}(g, o) \con \\
 \pr{organism\_lacks\_function}(o, f) &\imp & \n( \pr{has\_function}(g, f) )
\end{eqnarray*}
Example: \begin{eqnarray*}
 \pr{in\_organism\_type}( \pr{P35431},  \pr{Chicken}) \con \\
 \pr{organism\_lacks\_function}( \pr{Chicken},  \pr{Lactation}) &\imp & \n( \pr{has\_function}( \pr{P35431},  \pr{Lactation}) )
\end{eqnarray*}
\section{Phylogenetic Functional Inference Axioms}
Axioms for
inferring function for genes based on the ontology and the
phylogenetic tree. These are broken into two sub-sections: propagation
DOWN the phylogenetic tree, and propagation UP the phylogenetic
tree. From an operational point of view the first step would be to
propagate up to ancestral genes based on experimental evidence in
extant organisms, followed by propagation down to fill in the gaps in
extant genes.
\subsection{Down-propagation}
Inference of function of leaf node
genes. Inference DOWN the phylogenetic tree. We inherit genes from our
descendants, and also the function of those genes. Note that in order
to make any inferences, we first have to assign function to ancestral
genes.
\emph{Axiom: down-propagation: functions propagate down the phylogenetic; tree UNLESS there was an ancestral loss.  The results of this; inference can be used to populate species-specific GAFs. }

\begin{eqnarray*}
 \pr{descended\_from}(g, g_p) \con \\
 \pr{has\_function}(g_p, f) \con \\
 \n( \pr{lacks\_function}(g, f) )&\imp & \pr{has\_function}(g, f) 
\end{eqnarray*}
\emph{Axiom: down-propagation: functions propagate down the phylogenetic; tree UNLESS there was an ancestral gain}

\begin{eqnarray*}
 \pr{descended\_from}(g, g_p) \con \\
 \pr{lacks\_function}(g_p, f) \con \\
 \n( \pr{has\_function}(g, f) )&\imp & \pr{has\_function}(g, f) 
\end{eqnarray*}
\begin{eqnarray*}
 \pr{comment}( \pr{Thus far we have dealt with inferences that are all by
definition true. For the final }) 
\end{eqnarray*}
\subsection{Up-propgatation}
Inference of function of intermediate
node genes. Inference UP the phylogenetic tree. The assumption is that
if the descendant genes of an ancestral gene share the same function,
then either (a) the ancestor gene had that function or (b) the gene
evolved that function independently. The latter is more likely in the
case of gene duplications. We cannot reconstruct with 100%
accuracy. Ideally we would use a probabilistic model (eg Sifter)
combined with curator knowledge. The goal here with these logical
axioms is to present possibe candidate solutions.
\begin{eqnarray*}
 \pr{comment}( \pr{first we define two relations concerning the evolution and
loss of function}) 
\end{eqnarray*}
\subsubsection{Relation:  \pr{descendents\_independently\_evolved}}


G
descendents\_independently\_evolved F if there exists at least two
descendants of G that have function F, but F is not present in G.

\emph{Axiom: G has descendants that indepedently evolved F if and only if G; lacks F and G is the LCA of two genes that have F}

\begin{eqnarray*}
 \pr{has\_function}(d_1, f) \con \\
 \pr{has\_function}(d_2, f) \con \\
 \pr{least\_common\_ancestor}(d_1, d_2, g) \con \\
 \n( \pr{has\_function}(g, f) )&\dimp & \pr{descendents\_independently\_evolved}(g, f) 
\end{eqnarray*}
\subsubsection{Relation:  \pr{descendents\_independently\_lost}}


G
descendents\_independently\_lost F if there exists at least two
descendants of G that lack function F, and F is present in G.

\emph{Axiom: G has descendants that indepedently lost F if and only if G; lacks F and G is the LCA of two genes that have F}

\begin{eqnarray*}
 \pr{lacks\_function}(d_1, f) \con \\
 \pr{lacks\_function}(d_2, f) \con \\
 \pr{least\_common\_ancestor}(d_1, d_2, g) \con \\
 \pr{has\_function}(g, f) &\dimp & \pr{descendents\_independently\_lost}(g, f) 
\end{eqnarray*}
\subsubsection{Relation:  \pr{plausibly\_has\_function}}


G plausibly\_has\_function F if
there is reasonable grounds to believe that G has\_function F, based on
the descendants of G having function F

\emph{Axiom: when we say G plausibly\_has\_function F it means we are; guaranteed that either (a) it is indeed the case that G; has\_function F or (b) an orthologous pair descended from G have F,; and that they each evolved F independently. In practical terms, we; can present these inferences to a curator who can use judgement in; selecting the most likely scenario; or the results could be passed; to a probabilistic model}

\begin{eqnarray*}
 \pr{plausibly\_has\_function}(g, f) &\imp & \pr{has\_function}(g, f) \dis  \n( \pr{duplication}(g) )\con \\
&& \pr{descendents\_independently\_evolved}(g, f) 
\end{eqnarray*}
\emph{Axiom: when we say G plausibly\_lacks\_function F it means we are; guaranteed that either (a) it is indeed the case that G; lacks\_function F or (b) the descendants of G lack F, and that they; lost it independently. In practical terms, this can be used; to make NOT annotations to ancestral nodes, with care}

\begin{eqnarray*}
 \pr{plausibly\_lacks\_function}(g, f) &\imp & \pr{lacks\_function}(g, f) \dis  \pr{descendents\_independently\_lost}(g, f) 
\end{eqnarray*}
\emph{Axiom: propagation UP the phylogenetic tree:; here we would ideally have a probabilistic model. This; just gives *candidates* for realizing as curator-vetted; annotations}

\begin{eqnarray*}
 \pr{has\_function}(d_1, f) \con \\
 \pr{has\_function}(d_2, f) \con \\
 \pr{least\_common\_ancestor}(d_1, d_2, g) &\imp & \pr{plausibly\_has\_function}(g, f) 
\end{eqnarray*}
\emph{Axiom: propagation of negative annotations UP the phylogenetic; tree but DOWN the GO.}

\begin{eqnarray*}
 \pr{lacks\_function}(d_1, f) \con \\
 \pr{lacks\_function}(d_2, f) \con \\
 \pr{least\_common\_ancestor}(d_1, d_2, g) &\imp & \pr{plausibly\_lacks\_function}(g, f) 
\end{eqnarray*}
